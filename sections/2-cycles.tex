\section{Assenza di cicli}
L'assenza di cicli per particolari valori di $R$ può essere mostrata grazie al criterio di Bendixon.
\begin{equation}
    \textrm{div f}(x_1, x_2) = \frac{\partial f_1(x_1,x_2)}{\partial x_1} + \frac{\partial f_2(x_1,x_2)}{\partial x_2} = -\frac{R}{L} + \frac{\alpha}{C} - \frac{3\beta}{C}x_2^2
\end{equation}
\begin{equation}\label{big-resistance}
    R > \alpha \frac{L}{C} \Longrightarrow \textrm{div f}<0 \quad \forall x_2
\end{equation}
Per $R > \alpha \frac{L}{C} = 40 \Omega$, la divergenza assume valore negativo per qualunque $x_2$: dunque, non sono presenti cicli nell'intero piano.

Il risultato è in accordo con l'analogia intuitiva che associa la resistenza elettrica con l'attrito nei sistemi meccanici: entrambi i fenomeni dissipano energia e smorzano i movimenti periodici.
