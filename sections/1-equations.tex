\section{Equazioni di stato}

\begin{figure}
\centering
\begin{circuitikz}
\draw(0,0)
to[R, l=$R$, v_>=$v_R$] (0,4)
to[L, l=$L$, v_>=$v_L$, i<=$x_1$, -*] (4,4)
    node[label=A]{}
to[C, l=$C$, v=$x_2$, i>^=$i_C$] (4,0)
to (0,0);
\draw(4,4) to[short, i=$i$] (8,4)
to[generic, l=BNL, v=$x_2$] (8,0)
to (4,0);
\end{circuitikz}
\caption{Il circuito elettrico da studiare.}
\label{fig:circuito}
\end{figure}

Considero l'equazione caratteristica del condensatore $C$:
\begin{equation}
\label{cond-caratt}
i_c = C \dot{x_2}
\end{equation}

Applico LKC al nodo A:
\begin{equation}
i_c = -x_1 - i
\end{equation}

Sostituendo $i_c$ nell'equazione~\ref{cond-caratt}, ottengo la prima equazione di stato:
\begin{equation}
\dot{x_2} = -\frac{1}{C} x_1 + \frac{1}{C}\left( \alpha x_2 - \beta x_2^3\right)
\end{equation}

Poi, considero l'equazione caratteristica del condensatore $L$:
\begin{equation}
\label{ind-caratt}
v_L = L \dot{x_1}
\end{equation}

Applico LKT alla maglia di sinistra, ottenendo:
\begin{equation}
x_2 = v_R + v_L = x_1 R + L \dot{x_1}
\end{equation}

Da cui la seconda equazione di stato:
\begin{equation}
    \dot{x_1} = -\frac{R}{L}x_1 + \frac{1}{L} x_2
\end{equation}

Il sistema è dunque descritto dalle equazioni:
\begin{equation}
    \left\{
    \begin{aligned}
        \dot{x_1} =& -\frac{\strut R}{\strut L}x_1 + \frac{1}{L} x_2\\
        \dot{x_2} =& -\frac{\strut 1}{\strut C} x_1 + \frac{1}{C}\left( \alpha x_2 - \beta x_2^3\right)\\
    \end{aligned}
    \right.
\end{equation}
